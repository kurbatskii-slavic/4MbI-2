\documentclass[a4paper,12pt,titlepage,final]{article}

\usepackage[T1,T2A]{fontenc}     % форматы шрифтов
\usepackage[utf8x]{inputenc}     % кодировка символов, используемая в данном файле
\usepackage{amsfonts}
\usepackage{mathtools}
\usepackage[russian]{babel}      % пакет русификации
\usepackage{tikz}                % для создания иллюстраций
\usepackage{pgfplots}            % для вывода графиков функций
\usepackage{geometry}		 % для настройки размера полей
\usepackage{indentfirst}         % для отступа в первом абзаце секции
\usepackage{multirow}            % для таблицы с результатами
\usepackage{listings}
\lstset{extendedchars=\true}
% выбираем размер листа А4, все поля ставим по 3см
\geometry{a4paper,left=30mm,top=30mm,bottom=30mm,right=30mm}

\setcounter{secnumdepth}{3}      % включаем нумерацию секций и подсекций

\usepgfplotslibrary{fillbetween} % для изображения областей на графиках

\begin{document}
\begin{titlepage}
\centering\noindent
{
\begin{minipage}{0.1\textwidth}

\end{minipage}
\hfill
\begin{minipage}{0.77\textwidth}
\begin{center}
\textbf{МОСКОВСКИЙ ГОСУДАРСТВЕННЫЙ УНИВЕРСИТЕТ}\par
\textbf{имени М.В.Ломоносова}\par
\end{center}
\end{minipage}
\hfill
\begin{minipage}{0.1\textwidth}

\end{minipage}
}\par
{
\textbf{Факультет вычислительной математики и кибернетики}\par
\nointerlineskip
\noindent\makebox[\linewidth]{\rule{\textwidth}{0.4pt}}
}
\vfill
{
\Large{\textbf{Практическое задание по курсу лекций}}\par
\Large{\textbf{«Численные методы линейной алгебры»}}\par
}\\
{
\Large{\textbf{Задание №2}}\par
}
{
\Large{\textbf{Отчет}}\par
\Large{\textbf{о выполненном задании}}\par
\Large{студента 303 учебной группы факультета ВМК МГУ}\par
\Large{Курбацкого Вячеслава Константиновича}\par
}
\vfill
{\small Москва\\ \the\year{}}
\end{titlepage}
% Автоматически генерируем оглавление на отдельной странице
\tableofcontents
\newpage
\section{Прямые и итерационные методы решения СЛАУ}
\subsection{Постановка задачи}
Дана система линейных алгебраических уравнений:
$$x + Ax = F \Leftrightarrow (A + I)x = F$$
Причём матрица $A + I$ является симметричной и положительно определенной. Напомним, что матрица $B = B^T$ называется симметричной, а положительно определенной (обозначение $B > 0$) называется такая матрица, что $(Bx, x) > 0$ $\forall x \neq \theta$, где $\theta$ - нулевой вектор соответствующей размерности. Матрица определяется вариантом задания. \\
Требуется решить данную систему прямым методом, реализованным в задании 1, и итерационным методом Чебышёва. Так же предлагается оценить среднеквадратическую норму ($\Vert \cdot \Vert = \frac{1}{\sqrt{n}} \Vert \cdot \Vert_2$) ошибки и оценить спектр матрицы системы. Реализация алгоритмов осуществлена на языке $C++$.
\subsection{Описание методов}
\subsubsection{Процесс ортогонализации Грама-Шмидта}
Данный метод позволяет решать СЛАУ путём построения QR-разложения матрицы системы. Данный метод был подробно рассмотрен в задании 1 и реализация не претерпела никаких изменений.
\subsubsection{Метод Чебышева}
Как было заявлено выше, данный метод является итерационным. Отличие этих методов от прямых заключается в том, что в процессе решения СЛАУ строится итерационная последовательность векторов, сходящаяся к точному решению. Итерационные методы удобны тем, что можно повысить точность, увеличив число итераций алгоритма. В случае использования как прямых, так и итерационных методов точное решение никогда не будет найдено из-за ошибок округления и неточного представления вещественных чисел в виде чисел с плавающей точкой. Теперь подробней остановимся на одном из таких методов - методе Чебышева, рассмотренном и обоснованном на лекциях.

Пусть $A^T = A > 0$, $\lambda_{max}$, $\lambda_{min}$ - соответственно максимальное и минимальное собственное значение матрицы $A$. В силу положительно определенности и симметричности они будут вещественны и положительны. Обозначим:
$$\tau_0 = \frac{2}{\lambda_{max} + \lambda_{min}}, \quad
\kappa = cond(A) = \frac{\lambda_{max}}{\lambda_{min}}, \quad \rho_0 = \frac{1 - \kappa^{-1}}{1 + \kappa^{-1}}$$
Рассмотрим вектора $y^n$ такие, что:
$$\begin{cases}
 \frac{1}{\tau_n}(y^n - y^{n-1}) + Ay^{n-1} = f, \quad n = \overline{1, m} \\ \\
y^0 \: \text{- заданное начальное приближение}
\end{cases}$$
где
$$\tau_n = \frac{\tau_0}{1 + \rho_0cos\frac{\pi}{2m}(2n-1)}, \quad m - \text{число итераций метода Чебышева.}$$
Пусть также имеется точное решение $u^*$: $Au^* = f$. Введём вектор ошибки на каждой итерации: $z^n = y^n - u^*$. На лекции была доказана следующая теорема: \\ \\
\textbf{Теорема (о сходимости метода Чебышева).}  При всех указанных выше условиях и обозначениях:
$$\Vert z^m \Vert_2 \leq q_m \Vert z^0 \Vert_2, \text{ где}$$
$$q_m = \frac{2\rho_1^m}{1 + \rho_1^{2m}}, \quad \rho_1 = \frac{1 - \kappa^{-1/2}}{1 + \kappa^{-1/2}} < 1$$
То есть гарантируется сходимость метода к точному решению, т.к. $\Vert z^n \Vert_2 \to 0$. Метод Чебышева хоть и имеет намного более высокую скорость сходимости, чем, например, метод простых итераций, где имело место оценка $\Vert z^m \Vert_2 \leq \rho_0^m \Vert z^0 \Vert_2$, но в процессе реализации накладываются дополнительные трудности.\\\\
\textbf{Замечание.} Для вычислений по указанным выше формулам необходимо знать $\lambda_{max}$ и $\lambda_{min}$. На практике явное вычисление собственных значений у матриц большого размера не производится. Вместо этого достаточно найти $a, b:$ 
$$a \leq \lambda_{min} \leq \lambda_{max} \leq b$$
и использовать эти значения вместо $\lambda_{min},\: \lambda_{max}$. Точность от этого станет хуже (т.к. увеличится величина $\kappa$), но зато этот приём позволяет избежать явного вычисления спектра, ограничившись лишь некоторыми оценками на собственные значения матрицы.\\\\
\textbf{Замечание.} Метод Чебышева "заточен" \; под заранее выбранное число итераций $m$, т.к. оно напрямую используется в вычислениях параметров $\tau_n$. Это приводит к тому, что при появлении необходимости увеличить число итераций, придется заново пересчитывать все $\tau_n$. \\\\
\textbf{Замечание.} На лекции было показано, что на практике имеет место следующее явление: норма ошибки $\Vert z^n \Vert_2$ на некоторых итерациях увеличивается. При прямом порядке итераций $(n = 1, 2, \ldots m)$ такие "плохие" итерации будут следовать подряд, из-за чего ошибки округления будут стремительно расти, что может привести к переполнению до завершения финальной итерации. Таким образом, хоть сходимость и гарантированна теоретически, следует модифицировать алгоритм во избежании проблем с точностью на практике. Заданием предложено выбирать число итераций, равное степени двойки, то есть $n = 2^k, \; k \in \mathbb{N}$. Для такого случая на лекции был представлен алгоритм упорядочивания итераций, который и был использован в написанной программе. Пусть:
$$J^{(1)} = \{1\}, \quad J^{(2)} = \{1, 3\} $$
$$J^{(m)} = \{ j_1^{(m)},  j_2^{(m)}, \ldots j_m^{(m)}\} \Rightarrow$$
$$J^{(m)} = \{ j_1^{(m)}, \alpha_2^{(2m)} , j_2^{(m)}, \alpha_4^{(2m)} , \ldots j_m^{(m)}, \alpha_{2m}^{(2m)}\}, \text{ где}$$
$$\alpha_{2k}^{(2m)} = 4m - j_k^{(m)}$$
То есть данное реккурентное соотношение строит следующее множество индексов по предыдущему, расставляя все индексы из предыдущего множества на нечетные места и заполняя четные в соответствии с указанным соотношением. Такой выбор индексов можно интерпретировать следующим образом: чередуются "хорошие" итерации (уменьшающие норму ошибки) и "плохие" (увеличивающие эту норму). Сходимость от этого не перестанет иметь место, однако ошибка не будет накапливаться достаточно долго, чтобы вызвать переполнение.
\subsection{Оценка спектра матрицы}
Из курса линейной алгебры известна следующая теорема: \\ \\
\textbf{Теорема (Гершгорина).} Пусть $A \in \mathbb{C}^n$, введём следующие множества (круги Гершгорина):
$$\Gamma_i = \{z: |a_{ii} - z| \leq R_i\} ,\quad R_i = \sum_{j\neq i} |a_{ij}|$$
Тогда для любого собственного значения матрицы $\lambda$ верно:
$$\lambda \in \bigcup_{i=1}^n \Gamma_i$$ То есть все собственные значения лежат в объединении кругов Гершгорина. Стоит отметить, что можно считать радиусы, суммируя модули по столбцам. В данной задаче эта возможность не предоставит дополнительных оценок, т.к. матрица является симметричной. Пользуясь этой теоремой и учитывая, что элементы матрицы и её собственные значения вещественны, можно свести круги к отрезкам на вещественной прямой. В программе были найдены все круги, объединение которых даёт следующую оценку на собственные значения: $$\lambda \in [1, 158.6] \Rightarrow 1 \leq \lambda_{min} \leq \lambda_{max} \leq 158.6$$
То есть искомые параметры $a = 1$ и $b = 158.6$. Положив начальное приближение $y^0 = \theta$ (так предложено заданием), можно перейти к программной реализации метода Чебышева, в которой так же будет поиск наименьшего числа итераций, при которых погрешность решения на последней итерации в среднеквадратической норме не превосходит погрешность прямого метода.
\subsection{Программная реализация}
\subsubsection{Реализация метода Чебышева}
Нахождение решения осуществляется данной функцией: 
\begin{verbatim}
std::vector<double>
Chebyshev_solve(const Matrix& A, const std::vector<double>& f,
const std::vector<double>& y0, double a, double b, size_t m, 
const std::vector<size_t> &mask) // Chebyshev method
{
    double t0 = 2 / (a + b); // approximation for optimal tau
    double cond = b / a; // approximation for cond(A)
    double r0 = (1 - 1 / cond) / (1 + 1 / cond); // approximation for rho 
    double t_k;
    std::vector<double> y(y0);
    for (size_t k = 0; k < m; k++) { // iterations
        t_k = tau(t0, r0, m, mask[k]); // count parameter
        y = (f - A * y) * t_k + y; // update y
    }
    return y;
}
\end{verbatim}
со вспомогательной функцией $tau$,
\begin{verbatim}
double
tau(const double &t0, const double &r0, const size_t m, const size_t n)
{
    const double pi = 2 * std::acos(0);
    return t0 / (1 + r0 * std::cos(pi * (2n - 1) / (2 * m)));
}
\end{verbatim}
осуществляющей вычисление по формуле:
$$\tau_n = \frac{\tau_0}{1 + \rho_0cos\frac{\pi}{2m}(2n-1)}$$
Опишем параметры подробней:
\begin{itemize}
    \item $A$ - матрица системы. Класс матриц был реализован в прошлом задании. С полной реализацией можно ознакомиться в исходном коде.
    \item $f$ - правая часть уравнения.
    \item $y0$ - начальное приближение (в данном задании кладется равным нулю).
    \item $a$ - левая граница оценки спектра.
    \item $b$ - правая граница оценки спектра.
    \item $m$ - число итераций алгоритма
    \item $mask$ - оптимально упорядоченные индексы итераций ($J^{(m)}$)
\end{itemize}
\subsubsection{Результат работы прямого метода}

$$\begin{tabular}{|c|c|c|}
\hline 
Метод & $\Vert x - \Tilde{x} \Vert $ & $\Vert f - A\Tilde{x}\Vert$ \\
\hline
     Грам-Шмидт &  4.45256e-16 & 3.07939e-14 \\
\hline
\end{tabular}$$
Эти данные необходимы для нахождения числа итераций. Напомним, что заданием требуется найти минимальное число итераций, при котором норма ошибки в методе Чебышева будет не больше нормы ошибки прямого метода. Нахождение нужного числа итераций осуществлено перебором степеней 2, начиная с 1 степени.
\subsubsection{Нахождения числа итераций}
Данный цикл реализует вышеупомянутый перебор степеней двойки. Каждый раз пересчитывается множество $J^{(m)}$, и вычисляется решение с $count$ итерациями.
\begin{verbatim}
    std::vector<double> y0(SIZE, 0), y(SIZE, 0);
    size_t count = 1;
    std::vector<size_t> mask = {1};
    while (norm(x - y) >= x_error) {
        count *= 2;
        std::vector<size_t> new_mask(count);
        for (size_t k = 0; k < count; k++) {
            if (k % 2 == 0) {
                new_mask[k] = mask[k / 2];
            } else {
                new_mask[k] = 2 * count - mask[k / 2]; 
            }
        }
        mask = new_mask;
        y = Chebyshev_solve(A, f, y0, a, b, count, mask);
    }
\end{verbatim}
В результате выполнения программы получилось, что минимальное число итераций равняется 256.
\subsection{Норма ошибки метода Чебышева.}
В данной таблице и далее: $x$ - точное решение, $\Tilde{x} = \Tilde{x^m}$ - приближенное, $\Tilde{x^k}$ - приближенное решение на $k$-ой итерации.
$$\begin{tabular}{|c|c|c|c|c|}
\hline 
Метод & $\Vert x - \Tilde{x} \Vert $ & $\Vert f - A\Tilde{x}\Vert$  & $\frac{\Vert x - \Tilde{x} \Vert}{\Vert x \Vert}$ & Число итераций\\
\hline
     Грам-Шмидт & 4.45256e-16 & 3.07939e-14 &  7.4793e-16 & -\\
\hline
     Чебышев & 3.31777e-16 &  2.51709e-14 & 5.57311e-16 & 256 \\
\hline
\end{tabular}$$
Как и было заявлено, точность метода Чебышева выше, чем у прямого метода. Отметим, что точность ещё можно повысить, увеличив число итераций - в данном задании рассмотрено только наименьшее число итераций, для которого итерационный метод не хуже прямого. Так же важно уточнить, что от запуска к запуску минимальное число итераций может поменяться, т.к. вектор $x$ каждый раз генерируется случайно. Однако в большинстве случаев значения 256 достаточно.
\subsection{График $\Vert x - \Tilde{x}^k \Vert$ на каждой итерации}

\hspace{-3.3cm}
\includegraphics[scale=0.8]{images/figure.png}
Невооруженным взглядом видно, что практические результаты не противоречат теоретическим оценкам: погрешность действительно убывает немонотонно. Несмотря на периодические скачки, благодаря выбранному порядку норма не успевает вырасти достаточно, чтобы вызвать переполнение. Несмотря на немонотонность нормы ошибки, она всё равно сходится к нулю при увеличении числа итераций.
\section{Выводы}
Был реализован итерационный метод Чебышева решения СЛАУ. Для этого были оценены собственные значения матрицы с помощью теоремы о кругах Гершгорина. Количество итераций было подобрано минимальным таким, что итерационный метод имеет норму $\Vert z^m \Vert$ меньшую, чем прямой метод. Так же был построен график $\Vert z^k \Vert = \Vert x - \Tilde{x}^k \Vert$ в зависимости от номера итерации $k$.
\section{Инструкции по запуску программы.}
\begin{itemize}
    \item Запустить скрипт run.sh командой 
\begin{lstlisting}
$ ./run.sh 
\end{lstlisting} 
Данный скрипт создаст директорию build и внутри неё скомпилирует проект с помощью cmake и Makefile. 
    \item находясь в директории build команда 
\begin{lstlisting}
$ ./main 100 < ../tests/SLAU_var_6.txt
\end{lstlisting} 
    запустит программу на матрице размера $100 \times 100$, заданной вариантом. Кроме того, предусмотрена возможность тестирования на матрицах размера 1024 и 2048:
\begin{lstlisting}
$ ./main 1024 < ../tests/in1024.txt
$ ./main 2048 < ../tests/in2048.txt
\end{lstlisting} 
Во всех случаях вывод будет примерно такой:
\begin{lstlisting}
Gram-Shmidt:
||x - x_f|| = 4.45256e-16
||f - Ax_f|| = 3.07939e-14
||x - x_f|| / ||x|| = 7.4793e-16

Spectrum range:
Range: [1, 158.6]

Chebyshev:
||x - y|| = 3.31777e-16
||f - Ay|| = 2.51709e-14
||x - y|| / ||x|| = 5.57311e-16
Iterations count = 256
\end{lstlisting}
\end{itemize}

\end{document}